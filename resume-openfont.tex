%%%%%%%%%%%%%%%%%%%%%%%%%%%%%%%%%%%%%%%
% One Page Two Column Resume
% LaTeX Template
% Version 1.1 (16/9/2014)
%
% Author ; Kunaal Ahuja
%
% Source code forked from repository:
% https://github.com/deedydas/Deedy-Resume
%
% IMPORTANT: THIS TEMPLATE NEEDS TO BE COMPILED WITH XeLaTeX
%
% This template uses several fonts not included with Windows/Linux by
% default. If you get compilation errors saying a font is missing, find the line
% on which the font is used and either change it to a font included with your
% operating system or comment the line out to use the default font.
% 
%%%%%%%%%%%%%%%%%%%%%%%%%%%%%%%%%%%%%%


\documentclass[]{resume-openfont}
\usepackage{fancyhdr}
 
\pagestyle{fancy}
\fancyhf{}
 
\begin{document}

%%%%%%%%%%%%%%%%%%%%%%%%%%%%%%%%%%%%%%
%
%     TITLE NAME
%
%%%%%%%%%%%%%%%%%%%%%%%%%%%%%%%%%%%%%%
\namesection{}
{Kunaal Ahuja}{
\href{mailto:kunaal@gatech.edu}{kunaal@gatech.edu} || 
(678) 848-8960 ||
\href{http://www.kunaalahuja.com/}{kunaalahuja.com} ||
\href{https://www.linkedin.com/in/kunaal-ahuja/}{LinkedIn: kunaal-ahuja} ||  
\href{https://github.com/kunaalahuja/}{GitHub: kunaalahuja} \\
}
%%%%%%%%%%%%%%%%%%%%%%%%%%%%%%%%%%%%%%
%
%     COLUMN ONE
%
%%%%%%%%%%%%%%%%%%%%%%%%%%%%%%%%%%%%%%

\begin{minipage}[t]{0.33\textwidth} 

%%%%%%%%%%%%%%%%%%%%%%%%%%%%%%%%%%%%%%
%     EDUCATION
%%%%%%%%%%%%%%%%%%%%%%%%%%%%%%%%%%%%%%

\section{Education} 

\subsection{Georgia Institute of Technology}
\descript{Master of Science in Analytics}
\location{Computational Data Analytics}
\location{Aug'19 - Dec'20 | Atlanta, Georgia}
\location{CGPA : 4.0/4.0 | Completed all 3 tracks}
{\footnotesize \textit{\textbf{Teaching Assistant}}} \\
Computing for Data Analytics (Python)\\

\sectionsep

\subsection{NETAJI SUBHAS INSTITUTE OF TECHNOLOGY (NSIT)}
\descript{BS in Information Technology}
\location{2010 - 2014 | New Delhi, India}
\location{CGPA : 9.0/10 | Department Rank 2}
\sectionsep

%%%%%%%%%%%%%%%%%%%%%%%%%%%%%%%%%%%%%%
%     SKILLS
%%%%%%%%%%%%%%%%%%%%%%%%%%%%%%%%%%%%%%

\section{Skills}
\location{ANALYTICS AND LANGUAGES:}
Python and R for data analysis and machine learning \textbullet{}   C++ \textbullet{} Slang \textbullet{} Scala \\
Javascript \textbullet{} Matlab \textbullet{} D3 \textbullet{} \LaTeX  
\vspace{6pt}
\location{DATABASES and SOFTWARES:}
SQL (Sybase, MemSQL, MySQL, SQLite) \textbullet{} AWS S3 \textbullet{} Hive \textbullet{} Excel \textbullet{} Xpress \textbullet{} Tableau\vspace{6pt}
\location{DATA SCIENCE FRAMEWORKS:}
Reinforcement Learning \textbullet{Docker} \textbullet{PyTorch} \textbullet{} Hadoop \textbullet{} Spark \textbullet{} Tensorflow \textbullet{} HBase \textbullet{} MapReduce \textbullet{} D3.js \textbullet{} Flask
\sectionsep

%%%%%%%%%%%%%%%%%%%%%%%%%%%%%%%%%%%%%%
%     COURSEWORK
%%%%%%%%%%%%%%%%%%%%%%%%%%%%%%%%%%%%%%

\section{Coursework}
Deep Learning | Artificial Intelligence | 
Digital Marketing |
Natural Language | 
Design of Experiments | 
Machine Learning | 
Deterministic Optimization |
Data and Visual Analytics | 
Regression Analysis |
Operation Research | 
\sectionsep


%%%%%%%%%%%%%%%%%%%%%%%%%%%%%%%%%%%%%%
%     Research
%%%%%%%%%%%%%%%%%%%%%%%%%%%%%%%%%%%%%%

\section{Research}
\location{Ant based Clustering in MANETs}
Developed an efficient ant colony clustering algorithm in MANET. Created a probabilistic indicator to choose the cluster head based on the pheromone value \& its visibility in the network
\sectionsep

%%%%%%%%%%%%%%%%%%%%%%%%%%%%%%%%%%%%%%
%     AWARDS
%%%%%%%%%%%%%%%%%%%%%%%%%%%%%%%%%%%%%%
\section{Achievements}
\runsubsection{}
\vspace{\topsep} 
Awarded silver medal and academic scholarship for securing 2nd rank in college \\
\\\vspace{\topsep} 
Represented NSIT in the ACM ICPC South Asia regionals
\sectionsep


%%%%%%%%%%%%%%%%%%%%%%%%%%%%%%%%%%%%%%
%
%     COLUMN TWO
%
%%%%%%%%%%%%%%%%%%%%%%%%%%%%%%%%%%%%%%

\end{minipage} 
\hfill
\begin{minipage}[t]{0.66\textwidth} 

%%%%%%%%%%%%%%%%%%%%%%%%%%%%%%%%%%%%%%
%     EXPERIENCE
%%%%%%%%%%%%%%%%%%%%%%%%%%%%%%%%%%%%%%

\section{Work Experience}

\runsubsection{Cypress.io}
\descript{| Data Scientist Intern }
\location{May'20 - Aug'20 | Atlanta, GA}
\vspace{\topsep} 
\begin{tightemize}
\item Developed the data science platform and built prediction models for the subscription of the new clients and churn and upgrades of the existing clients leading to a 7\% MOM increase in subscriptions and reduced churn rate to 1\%.
\item Built the data science infrastructure to deploy models, implement version control in the data and run the jobs on the AWS cloud infrastructure
\item Refined the pricing & financial model for the optimized plans and KPI sales funnel
\item Designed the data science tools - A/B testing significance calculator, size calculator and reporting tools for the VC, board members, and management 
\end{tightemize}

\subsection{Goldman Sachs}
\descript{Data Scientist, Associate, Fixed Income Desk, GSAM}
\location{Aug 2017 - July 2019 | Bangalore, India}
\begin{tightemize}
\item Developed statistical models to predict the prices of assets and help portfolio managers on multi billion dollar portfolio construction
\item Revamped the multi factor risk model used for budgeting and risk management by adding granularity to the factors. Built the multi environment research platform to support by creating a bridge between Python and Slang. 
\end{tightemize}

\descript{Data Engineer, Data Platform}
\location{June 2014 - July 2017 | Bangalore, India}
\begin{tightemize}
\item Designed the database layer (HDFS, Hive) to support big-data solutions, the compute layer comprising of Spark for transformation and the serving layer to fetch data with low latency (improved from a minute to sub-second)
\item Designed the caching infrastructure to store the realtime data for financial products to enhance the performance of the backtest engine for systematic strategies. This reduced the running time of models from 1 hour to 10 min
\end{tightemize}
\sectionsep

%%%%%%%%%%%%%%%%%%%%%%%%%%%%%%%%%%%%%%
%     RESEARCH
%%%%%%%%%%%%%%%%%%%%%%%%%%%%%%%%%%%%%%

\section{Academic Projects}

\runsubsection{YELP Review Question Answering Framework}
\descript{}\location{Jan'20 - May'20 | \href{https://github.com/kunaalahuja/YelpQuestionAnswering}{GitHub} | \href{http://www.kunaalahuja.com/YelpQuestionAnswering/}{Webpage}}
\begin{tightemize}
\item Developed a question-answering framework for user queries using BERT model trained with SQuaD and the set of classified genuine reviews 
\item Built a ensemble classifier to detect fake reviews on Yelp by using models like CNN, Gradient Boosting, BERT to classify the reviews with 88\% accuracy.
\end{tightemize}
\sectionsep

\runsubsection{Detecting Duplicate Quora Questions}
\descript{}\location{Jan'20 - May'20 | \href{https://github.com/kunaalahuja/DuplicateQuoraQuestions}{GitHub} | \href{https://github.com/kunaalahuja/DuplicateQuoraQuestions/blob/master/Project\%20Report.pdf}{Project Report}}
\begin{tightemize}
\item Built a NLP model to identify similar questions using an XGBoost and AdaBoost ensemble model of BERT, Attention based LSTMs, Siamese LSTMs.
\end{tightemize}
\sectionsep

\runsubsection{SNOTRA - Book Recommendation Application}
\descript{}\location{Aug'19 - Dec'19 | \href{https://github.com/kunaalahuja/DuplicateQuoraQuestions}{GitHub} | \href{https://github.com/kunaalahuja/SNOTRA/blob/master/SNOTRA\%20-\%20poster.pdf}{Poster Presentation}}
\begin{tightemize}
\item Developed a hybrid recommendation engine using content based and collaborative filtering for books with an accuracy of 86\%
\item Used cosine similarity and KNN normalized ratings for the  user-to-user collaborative and content based filtering on the books using TF-IDF approach
\item Built an interactive web application using Flask and  D3/javascript to display force layout graphs between similar users and books
\end{tightemize}
\sectionsep

\runsubsection{Mortality Rate in United States}
\descript{}\location{Aug'19 - Dec'19 | \href{https://github.com/kunaalahuja/Mortality-Model}{GitHub} | \href{https://github.com/kunaalahuja/Mortality-Model/blob/master/mortality_report.pdf}{Project Report}}
\begin{tightemize}
\item Built a \emph{mixed effect regression model} to predict the expected age at the time of death of various races in the United States
\item Discovered the demographic factors that may impact age at the time of death
\end{tightemize}
\sectionsep

% \runsubsection{Ant based Clustering Algo in adhoc Networking}
% \descript{}\location{}
% \begin{tightemize}
% \item Research project on developing an efficient ant colony clustering algorithm in MANET. Created a probabilistic indicator to choose the cluster head based on the pheromone value \& its visibility in the network
% \end{tightemize}
% \sectionsep

\end{minipage} 
\end{document}  \documentclass[]{article}